\chapter{引言}
\label{cha:intro}

\section{研究背景}
\label{sec:01_general_intro}

互联网早期网络环境简单,设备带宽资源有限,基于Best-effort服务原则并受限于滑动窗口流量控制机制和TCP拥塞控制算法的网络协议栈足够满足当时需求。然而随着移动互联网的快速发展和人工智能海量数据处理中底层大规模分布式机器学习框架的兴起,不仅数据中心的网络流量飞速提高,而且对网络处理的延迟和吞吐都提出更苛刻的要求。在当今短视频、微博、游戏等各种业务的蓬勃发展,网络流量中HTTP、RPC等常见通信协议存在着大量的TCP短连接已经让网络协议栈有些不堪重负,TCP短连接意味着三次握手、四次挥手这些复杂的TCP交互仅传递少量有效网络数据。根据相关调查显示,在较大规模的蜂窝网络中,有达到90\%的TCP流量是小于32KB,甚至小于4KB的TCP流也有超过一半之多~\cite{woo2013comparison},这无疑对数据中心中宝贵的CPU资源造成巨大浪费。

与此同时,在硬件设备方面,随着40Gb、100Gb乃至200Gb带宽的硬件网卡设备逐渐部署,以及CPU经历摩尔定律的主频指数增加之后又开始往多核架构方向发展,并且DDR3、DDR4高速物理内存和固态硬盘等设备逐渐应用在高速网络数据中心中,迭代缓慢的传统内核网络协议栈已经难以跟上硬件设备的更新以及网络流量的飞速发展,这使得网络性能瓶颈的压力又转移到软件层面的传统内核网络协议栈上。

所以,加速网络协议栈就成为如今整体提升网络性能的重中之重。最近几年在该领域的研究可以大体归为三类。第一类研究方向是将网络协议栈向下offload到硬件设备网卡,比如通过RDMA网卡进行彻底地绕过CPU和内核从而实现极致的网络性能提升,但此路径依赖硬件设备并且仅适用于超高网络性能要求的专用场景,完全不具备对网络应用的兼容性;第二类在内核态对现有内核网络协议栈进行调优,该方式由于无法完全避免系统调用带来的开销并且难以利用高速IO收发包框架从而造成网络性能提升有限;第三类是将协议栈完全搬离到用户态来实现,这在绕过内核和高速IO框架等优化方式下确实可获得显著的性能提升,不过移植传统网络应用通常还需要修改应用源码。这三条路径本殊途同归,最终都是为了减少网络协议栈对整个系统的开销从而提升网络应用的服务性能指标,然而过分追求网络性能的提升而忽略协议栈本身的兼容性会导致移植传统网络应用成本过高等兼容性问题。此外,由于在硬件网卡中加速网络协议栈初衷仅适用于专用场景,所以我们在后续第二章中仅对后两条路径做更加详细的说明。

TCP/IP协议栈最初作为操作系统的一部分,在Unix、Linux、Solaris、FreeBSD等各个系统中都有实现,为了能让网络应用可以直接运行在众多系统,各个网络协议栈向上层网络应用暴露服务接口时候都必须符合POSIX标准,这为后来互联网的飞速发展提供了扎实的技术基础。近几年影响力较大的用户态高性能网络协议栈mTCP~\cite{mTCP}由于移植网络应用需修改其源码导致从发布至今四年多时间里只成功移植Lighttpd这一款主流的网络服务器应用,而内核态网络协议栈Fastsocket~\cite{fastsocket}虽然性能提升没有mTCP明显但因具备高度应用兼容性已经部署到新浪微博的生成环境中。从中不难看出兼容性对高性能网络协议栈的后续发展至关重要。

在上述背景下,本文在用户态设计并实现了一款基于高速轮询IO DPDK~\cite{DPDK}框架的网络协议栈,并通过LD\_PRELOAD动态链接技术、文件描述符空间重映射机制和并发流CPU资源调度模块等解决种类繁多的网络模型的兼容性问题,在保证用户态协议栈网络性能提升明显的前提下,实现了高度兼容性,即传统网络应用在不需要修改任何源码的情况下将可以直接移植到该协议栈上获得网络性能的提升,几乎达到移植零成本。

\section{研究问题及其意义}
\label{sec:01_problem_intro}
本文针对当前网络协议栈的兼容性普遍较差、需要修改网络应用源码才能获得网络性能提升的问题,设计并在用户态实现一款兼具高性能的同时可以完全不需要修改网络应用源码即可完成移植的高度兼容的网络协议栈,这样为开发人力资源紧张的中小型企业几乎零成本地带来网络性能的提升。本文重点要解决的问题是如何完成对多进程、多线程、Epoll IO多路复用等网络编程模型丰富的多种应用同时进行支持,并且要保证能在不需要修改网络应用源码即可完成移植。此外,网络协议栈想要推广也必须有较为明显的网络性能提升,这就需要协议栈在设计之初就利用高速IO收发包框架绕过内核、并在多个进程或线程同时运行的情况下合理地完成对CPU资源的调度与分配,通过合理且高效的数据结构来完成进程间通信。

当前学术界已经有不少高性能用户态协议栈为网络应用带来显著的性能提升~\cite{mTCP,IX,ZygOS},然而过度追求高性能,尤其是向网络应用暴露硬件网卡收发包的控制细节虽然会带来一定的性能提升,但导致协议栈的接口脱离传统的POSIX API语义,对网络应用开发人员十分不友好。这种高性能用户态协议栈若在某种专用场景下重新开发对网络性能指标要求极高的应用较为适用,但是想将其推广到各种传统网络应用对开发人员将是十分棘手的事情。

所以对网络应用高度兼容的网络协议栈的意义十分明确,就是可以不需要修改网络应用源码的前提下,零成本地完成对多进程、多线程、IO多路复用等各种主流网络应用的移植,并带来较为明显的网络性能提升。

\section{研究难点与挑战}
\label{sec:01_difficulties_and_chanllenge}
本文的研究难点主要分为以下几个方面:

第一,就是如何在不修改传统网络应用的源码的前提下完成对多种网络编程模型的应用的支持,其中既要支持常见的socket、bind、listen、accept、read、write等网络函数,也要支持epoll这种高效的IO多路复用机制,还要对shutdown、dup、dup2等这些并不常见但也偶尔出现于主流网络应用的接口进行支持。与此同时,完成对这些与POSIX API系统调用的劫持,并能最大程度保留read、write、epoll\_wait这些函数对非网络文件描述符的支持。

第二,在用户态实现网络协议栈的同时并保证协议栈功能的完整性、鲁棒性与安全性,本文基于Linux内核3.14.2版本将其网络协议栈功能部分剥离出来,这需要在熟练掌握内核协议栈实现的基础上去划分网络功能的边界,并用高速IO收发包模块的底层开发套件替换掉内核协议栈中网卡netdev、内存管理、定时器等边界函数实现,具有较高的技术挑战。

第三,协议栈运行过程中可能存在多个进程或线程同时运行于系统中,包括协议栈进程、网络应用进程等,如何将这些并发流在有限的CPU核资源上合理进行分配与调度,在尽可能最大化利用CPU资源的同时尽量减少或避免对CPU资源的严重竞争,这对于多进程应用的网络性能也尤为重要,并且在用户态完成该工作也具备一定技术挑战。

\section{论文结构安排}
本文由六个章节组成,主要内容分别如下。

第一章概括性地介绍高性能网络协议栈兼容性的研究背景,并指出本文的研究内容、研究意义与研究挑战。

第二章从内核态和用户态两个方面介绍高性能协议栈的相关研究工作,并详细地介绍几款经典的高性能协议栈的系统架构与实现以及它存在的问题,并与本文工作进行对比分析,突出本工作的研究挑战性与意义。

第三章介绍该用户态协议栈在兼容性方面的设计,其中会先对协议栈系统的整体架构进行介绍,包括协议栈与网络应用的分核设计、IO多路复用接口的事件通知机制的设计与进程间通信的设计,接下来会重点介绍协议栈在兼容性方面的设计,其中包括通过LD\_PRELOAD动态链接技术完成对POSIX系统调用的劫持、文件描述符空间重映射设计完成对网络文件描述符与非网络文件描述符的区分、并发流CPU资源的调度与管理、以及系统异常崩溃时候进行资源回收等设计,从而实现网络应用的高度兼容性。

第四章对该用户态协议栈兼容性方面的实现进行详细介绍,包括进程间通信消息结构的设计,socket和文件描述符等关键资源的数据结构的设计与优化以及各种网络接口的具体实现方案。

第五章对已经成功移植的主流网络应用Nginx、Lighttpd、Redis以及独立开发的一款多线程Epoll网络服务器进行实验性能方面的评估并与最新版本的Linux内核进行对比。

第六章对本文的研究工作成果进行总结,并为网络协议栈兼容性更进一步的设计提供展望。

