\thusetup{
  %******************************
  % 注意:
  %   1. 配置里面不要出现空行
  %   2. 不需要的配置信息可以删除
  %******************************
  %
  %=====
  % 秘级
  %=====
  secretlevel={秘密},
  secretyear={10},
  %
  %=========
  % 中文信息
  %=========
  ctitle={高性能用户态协议栈的\\兼容性研究},
  cdegree={工程硕士},
  cdepartment={计算机科学与技术系},
  cmajor={计算机技术},
  cauthor={姜惠友},
  csupervisor={李~~~丹副教授},
  % 日期自动使用当前时间,若需指定按如下方式修改:
  % cdate={超新星纪元},
  %
  % 博士后专有部分
  cfirstdiscipline={计算机科学与技术},
  cseconddiscipline={系统结构},
  postdoctordate={2009年7月——2011年7月},
  id={编号}, % 可以留空: id={},
  udc={UDC}, % 可以留空
  catalognumber={分类号}, % 可以留空
  %
  %=========
  % 英文信息
  %=========
  etitle={Research on Compatibility of High-performance Userspace Protocol Stack},
  % 这块比较复杂,需要分情况讨论:
  % 1. 学术型硕士
  %    edegree:必须为Master of Arts或Master of Science(注意大小写)
  %             “哲学、文学、历史学、法学、教育学、艺术学门类,公共管理学科
  %              填写Master of Arts,其它填写Master of Science”
  %    emajor:“获得一级学科授权的学科填写一级学科名称,其它填写二级学科名称”
  % 2. 专业型硕士
  %    edegree:“填写专业学位英文名称全称”
  %    emajor:“工程硕士填写工程领域,其它专业学位不填写此项”
  % 3. 学术型博士
  %    edegree:Doctor of Philosophy(注意大小写)
  %    emajor:“获得一级学科授权的学科填写一级学科名称,其它填写二级学科名称”
  % 4. 专业型博士
  %    edegree:“填写专业学位英文名称全称”
  %    emajor:不填写此项
  edegree={Master of Engineering},
  emajor={Computer Technology},
  eauthor={Jiang Huiyou},
  esupervisor={Professor Li Dan},
  % eassosupervisor={Chen Wenguang},
  % 日期自动生成,若需指定按如下方式修改:
  % edate={December, 2005}
  %
  % 关键词用“英文逗号”分割
  %ckeywords={\TeX, \LaTeX, CJK, 模板, 论文},
  %ekeywords={\TeX, \LaTeX, CJK, template, thesis}
}

% 定义中英文摘要和关键字
\begin{cabstract}
随着移动互联网和人工智能的迅猛发展,不仅网络数据中心的整体流量与日俱增,而且由于RPC、HTTP、KV存储等业务的重要性不断提升导致小包数据占整个网络流量的比重越来越高。此外,随着硬件设备的迭代升级,40Gb、100Gb等高吞吐硬件网卡已经逐渐在数据中心广泛部署,传统网络内核协议栈作为底层网络传输网卡与上层网络应用之间的沟通桥梁,开始严重拖慢数据中心的网络性能。近几年工业界和学术界主要有在内核态和用户态对网络协议栈进行优化的两种研究途径,在内核态调优协议栈由于无法完全避免系统调用带来的开销造成性能提升有限,而在用户态实现的高性能协议栈普遍存在着兼容性差、移植成本高的问题,至今依然没有出现一款在保证网络性能明显提升的前提下达到网络应用高度兼容的协议栈系统。

本文针对当前高性能网络协议栈普遍存在着传统应用移植代价高、编程接口复杂的问题,在高性能用户态协议栈的兼容性方面进行研究、设计与实现,使得传统应用在完全不需要修改任何源码的情况下就能得到网络性能的提升。该协议栈基于DPDK高速轮询IO收发包框架,并将Linux内核网络协议栈源码搬离到用户态实现,通过LD\_PRELOAD动态链接技术、文件描述符空间重映射机制和并发流CPU资源调度模块等解决种类繁多的网络模型的兼容性问题,同时协议栈进程与应用进程采用分核设计并分别进行分组绑定、绕过内核减少系统调用等设计也让网络应用兼具用户态协议栈的高性能优势。在兼容性方面,本协议栈既支持常见的POSIX网络API和Epoll IO多路复用相关接口,又能支持read、epoll\_wait等相关接口对非网络文件描述符的调用,还能兼容多进程、多线程等各种主流网络编程模型。此外开发了对协议栈进程和网络worker子进程进行监控的模块,当进程异常退出时候能回收文件描述符、socket结构体等资源,提升协议栈系统的鲁棒性。

Nginx、Lighttpd、Redis等主流网络应用已经被移植到该用户态协议栈。实验结果表明,在完全不需要修改源码的前提下,该协议栈为Nginx、Lighttpd应用带来30\%~60\%网络吞吐上的提升,减少近一半的平均延迟,同时将尾延迟大幅降低到原来的四十分之一。


\end{cabstract}

% 如果习惯关键字跟在摘要文字后面,可以用直接命令来设置,如下:
\ckeywords{兼容性;用户态协议栈;高性能}

\begin{eabstract}
With the rapid development of mobile Internet and artificial intelligence, not only the overall network traffic in data centers is increasing, but also the proportion of small packet in the entire network traffic soars due to the growing importance of services such as RPC, HTTP, and KV storage. On top of that, with the continual upgrade of hardware devices, 40Gb, 100Gb and other high-throughput NICs have been widely deployed in the data center. The traditional kernel protocol stack serving as a bridge between the underlying network transmission medium and the upper-layer network applications begins to seriously slow down network performance in the data center. In recent years, there are two research approaches in the industry and academia to optimize the network protocol stack, either in kernel space or user space. The performance improvement of optimization in kernel space is limited due to the inability to completely avoid system calls. And the user-space high-performance protocol stacks generally are poor in compatibility and high at transplantation cost. So far, there is no protocol stack system with both high-compatibility and high-performance.

In this paper, in order to address the issues of high transplantation cost and complicated programming interface in the state-of-the-art network protocol stack, the compatibility of high-performance user-space protocol stack has been researched, designed and implemented, that is, without any modification of source codes legacy applications can attain significantly improved network performance. The protocol stack is based on the DPDK high-speed polling IO packet processing framework, and shifts up the Linux kernel source codes of network function to the user space. The compatibility issue of variety in network programming models has been solved by LD\_PRELOAD dynamic link technology, file descriptor space remapping mechanism and concurrent stream CPU resource scheduling module. In addition, with seperation design between protocol stack process and application process, bypassing the kernel to reduce system calls, the high-compatibility protocol stack also retain the advantage of high performance in user-space. In terms of compatibility, the protocol stack supports both common POSIX network APIs and Epoll IO multiplexing related interfaces, as well as read, epoll\_wait and other related interfaces with non-network file descriptor input parameter, and it is compatible with multi-processes and multi-threads of mainstream applications. In addition, a module for monitoring the protocol stack process and the application process has been developed. When the application process exits abnormally, resources such as file descriptors and socket structures can be recycled, and the robustness of the protocol stack system has been improved.

Major network applications such as Nginx, Lighttpd, and Redis have been transplanted onto the user-space protocol stack. The experimental results show that with no modification of source codes, the protocol stack brings 30\%$\sim$60\% network throughput improvement for Nginx and Lighttpd Web applications, cuts down the average latency by nearly half, and dramatically reduces the tail latency to fortieth.

\end{eabstract}

\ekeywords{compatibility; user-space protocol stack; high performance}
